\documentclass[A4paper,11pt]{article}
\usepackage[utf8]{inputenc}
\usepackage{graphicx}
\usepackage{geometry}
\usepackage[dvipsnames]{xcolor}
\usepackage{hyperref}
\usepackage{diagbox}
\usepackage{amsmath}
\usepackage{wrapfig}
\usepackage{listings} %code highlighter
\usepackage{color} %use color
\usepackage{subcaption}
\usepackage{float}
\usepackage[french]{babel}
\usepackage{comment}


\geometry{
 a4paper,
 total={150mm,240mm},
 left=20mm,
 top=20mm,
 right=20mm
 }

\title{Mathématiques Discrètes : Examen de Juin 2022}
\author{Marco Saerens, retranscrit par Doeraene Anthony}
\date{June 2022}

\begin{document}

\maketitle

\section{Logique}
\begin{enumerate}
    \item Soit la formule $p\Rightarrow (q\Rightarrow (p\land q))$. 
    \begin{enumerate}
        \item Cette formule est-elle une tautologie ou une contradiction? Montrez-le via une table de vérité
        \item Montrez-le via un raisonnement formel
    \end{enumerate}
    
    \item Soit les affirmations suivantes en logique des prédicats avec quantificateur
    \begin{itemize}
        \item Tous les chiens aiment les êtres humains ou détestent les chats
        \item Buzzy est un chien
        \item Buzzy aime les chats
    \end{itemize}
    \begin{enumerate}
        \item Appliquez les règles d'inférence pour tirer une conclusion
        \item Quelle est la conclusion que l'on peut tirer à partir de ces affirmations (en logique+en français)?
    \end{enumerate}
    \item Qu'est ce qu'une preuve par contradiction? Donnez une explication de ce concept ainsi la logique se trouvant derrière celle-ci.
\end{enumerate}

\section{Combinatoire}
\begin{enumerate}
    \item \begin{enumerate}
        \item Soit l'expression $(\sqrt{2}x - y )^5$. Donnez la forme développée de ce polynôme.
        \item Soit l'expression $(ax + by)^c$. Donnez le coefficient numérique de $x^{c-d}y^d$
    \end{enumerate}
    \item Nous possédons un générateur de mot de passe, pouvant utiliser comme caractère les lettres minuscules ainsi que les chiffres (0-9). Les lettres majuscules sont donc inutilisées. En supposant que ce générateur est parfaitement aléatoire (chaque caractère a la même probabilité d'apparaître). Quelle est la probabilité que le mot de passe
    \begin{enumerate}
        \item  commence par une lettre \textbf{et} termine par une lettre?
        \item  commence par une lettre \textbf{ou} termine par une lettre.
        \item ne contienne pas deux fois le même caractère?
        \item alterne les chiffres et les lettres (après chaque chiffre, nous avons une lettre et inversément)?
    \end{enumerate}
    
    \item Vous êtes actionnaire et souhaitez placer des actions parmis 5 entreprises différentes.
    \begin{enumerate}
        \item Combien y-a-t il de manière de placer 10 actions parmis ces 5 entreprises?
        \item Un ami vous recommande de répartir les actions afin de garantir une certaine rentrée. Vous devez donc répartir 20 actions entre 5 entreprises en s'assurant que chaque entreprise contienne au moins 2 actions.
    \end{enumerate}
\end{enumerate}

\section{Graphes}
\begin{enumerate}
    \item GRAPHE HERE\begin{enumerate}
    \item Trouvez la longueur du plus court chemin dans le graphe 
    \item Donnez le plus court chemin du graphe. Donnez également le plus court chemin entre le noeud a et le noeud h.
    \end{enumerate}
    \item Le graphe possède-t-il un circuit et/ou un chemin eulérien
    \item Dérivez la formule de calcul du score PageRank à partir de l'interprétation du \textit{random walker}
\end{enumerate}

\section{Equations de récurrence}
\begin{enumerate}
    \item Soit l'équation de récurrence homogène 
    \[ a_n = -3a_{n-1} + 10a_{n-2}\]
    \begin{enumerate}
        \item Donnez les racines de l'équation caractéristique associée
        \item Donnez la solution générale de cette équation
        \item Déterminez les coefficients avec les conditions initiales $a_0 = 0$ et $a_1 = 2$
    \end{enumerate}
    \item Nous rajoutons désormais un terme de source $F(n) = - 3^n$, arrivant donc à l'équation non-homogène suivant
    \[ a_n = -3a_{n-1} + 10a_{n-2} - 3^n\]
    \begin{enumerate}
        \item Démontrez la forme de la solution à une équation de récurrence non-homogène
         \item Donnez la solution générale de cette équation non-homogène
    \end{enumerate}
\end{enumerate}

\begin{comment}
Soit l'équation de récurrence linéaire de degré k non-homogène avec terme de source:
\[
a_n = a_{n-1} + a_{n-2} + ... + a_{n-k} + F(n)
\]
\begin{align*}
    &F(n) = \alpha_1n^m + \alpha_2n^{m-1} + ... + \alpha_{m}n + \alpha_{m+1} => a_n = \beta_1n^m + \beta_2n^{m-1} + ... + \beta_{m}n + \beta_{m+1}\\
    &F(n) = s^n \textrm{ et s n'est pas racine} => a_n = cs^n\\
    &F(n) = s^n \textrm{ et s est racine de multiplicité m} => a_n = cn^ms^n\\
    &F(n) = (\alpha_1n^m + \alpha_2n^{m-1} + ... + \alpha_{m}n + \alpha_{m+1})s^n \textrm{ et s n'est pas racine} \\ &=> a_n = (\beta_1n^m + \beta_2n^{m-1} + ... + \beta_{m}n + \beta_{m+1})s^n\\
    &F(n) = (\alpha_1n^m + \alpha_2n^{m-1} + ... + \alpha_{m}n + \alpha_{m+1})s^n \textrm{ et s est racine de multiplicité m} \\ &=> a_n = (\beta_1n^m + \beta_2n^{m-1} + ... + \beta_{m}n + \beta_{m+1})n^ms^n
\end{align*}


\begin{align*}
a_n &= an1^n\\
a_n &= 3a_{n-1} - 2a_{n-2} + 1\\
an1^{n} &= 3a(n-1)1^{n-1} -2a(n-2)1^{n-2} + 1\\
an &= 3an -3a -2an + 4a +1\\
0 &= a + 1\\
a&=-1
\end{align*}

$\exists! xP(x) \equiv \exists x(P(x)\land \forall y (P(y) \Rightarrow x=y))$

$\neg (p\Rightarrow q) \equiv \neg (\neg p \lor q) \equiv p \land \neg q$
\end{comment}
\end{document}
